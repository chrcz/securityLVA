%%%%%%%%%%%%%%%%%%%%%%%%%%%%%%%%%%%%%%%%%%%%%%%%%%%%%%%%%%%%%%%%%%%%%%
%
% Institut für Rechnergestuetzte Automation
% Forschungsgruppe Industrial Software
% Arbeitsgruppe ESSE
% http://security.inso.tuwien.ac.at/
% lva.security@inso.tuwien.ac.at
%
% Version 2013-10-24
% 
%%%%%%%%%%%%%%%%%%%%%%%%%%%%%%%%%%%%%%%%%%%%%%%%%%%%%%%%%%%%%%%%%%%%%%

\documentclass[12pt,a4paper,titlepage,oneside]{scrartcl}
\usepackage{esseProtocol}

%%%%%%%%%%%%%%%%%%%%%%%%%%%%%%%%%%%%%%%%%%%%%%%%%%%%%%%%%%%%%%%%%%%%%%
%
% FOR STUDENTS
%
%%%%%%%%%%%%%%%%%%%%%%%%%%%%%%%%%%%%%%%%%%%%%%%%%%%%%%%%%%%%%%%%%%%%%%

% Group number or "0" for Lab0
\newcommand{\gruppe}{21}
% Date
\newcommand{\datum}{2013-11-28}
% valid values: "Lab0", "Lab1" (be sure to use Uppercase for first character)
\newcommand{\lab}{Lab1}

% name of course
\newcommand{\lvaname}{Introduction to Security}
% number of course
\newcommand{\lvanr}{183.594}
% year and term, for example: "SS 2012", "WS 2012", "SS 2013", etc.
\newcommand{\semester}{WS 2013}

% Student data in Lab0 or 1. student of group in Lab1
\newcommand{\studentAName}{Christoph Czurda}
\renewcommand{\studentAMatrnr}{0704272}
\newcommand{\studentAEmail}{e0704272@student.tuwien.ac.at}

% 2. student of group in Lab1, for Lab0 or if your group has less students, remove these 3 lines
\newcommand{\studentBName}{Alexander Karla-Hager}
\renewcommand{\studentBMatrnr}{0925055}
\newcommand{\studentBEmail}{e0925055@student.tuwien.ac.at}

% 3. student of group in Lab1, for Lab0 or if your group has less students, remove these 3 lines
\newcommand{\studentCName}{Rainer Schwarzinger}
\renewcommand{\studentCMatrnr}{9027586}
\newcommand{\studentCEmail}{e9027586@student.tuwien.ac.at}

% 4. student of group in Lab1, for Lab0 or if your group has less students, remove these 3 lines
\newcommand{\studentDName}{Philipp Schiffner}
\renewcommand{\studentDMatrnr}{0925766}
\newcommand{\studentDEmail}{e0925766@student.tuwien.ac.at}

%%%%%%%%%%%%%%%%%%%%%%%%%%%%%%%%%%%%%%%%%%%%%%%%%%%%%%%%%%%%%%%%%%%%%%
%
% DO NOT CHANGE THE FOLLOWING PART
%
%%%%%%%%%%%%%%%%%%%%%%%%%%%%%%%%%%%%%%%%%%%%%%%%%%%%%%%%%%%%%%%%%%%%%%

\newcommand{\lang}{de}
\newcommand{\colormode}{color}
\newcommand{\dokumenttyp}{Abgabedokument \lab}

\begin{document}

\maketitle
\setcounter{section}{0}
\setcounter{tocdepth}{2}
\tableofcontents

%%%%%%%%%%%%%%%%%%%%%%%%%%%%%%%%%%%%%%%%%%%%%%%%%%%%%%%%%%%%%%%%%%%%%%
%
% CONTENT OF DOCUMENT STARTS HERE
%
%%%%%%%%%%%%%%%%%%%%%%%%%%%%%%%%%%%%%%%%%%%%%%%%%%%%%%%%%%%%%%%%%%%%%%

\section{Task 1}

\subsection{Welche Firma hat den Auftrag geschickt?}

WEE

\subsection{Vorgehensweise}

Man musste vom ASCII Code jedes Zeichens 4 abziehen und wenn man unter 65 (Wert von 'A') kommt, wieder von oben anfangen, also von 90 (Wert von 'Z') abziehen. Dann ergibt sich in der letzten Zeile der Firmenname WEE.

\section{Task 2A}

\subsection{Wie lautet der WEP Schlüssel?}

00:00:0D:EA:DB:EE:FD:EA:DB:EE:F0:00:00

\subsection{Vorgehensweise}

\begin{lstlisting}
aircrack-ng i_dump01_eqM4mptgSayoYpd8.cap
\end{lstlisting}

\section{Task 2B}

\subsection{Wie lautet der WPA-Schluessel?}

semisecret

\subsection{Vorgehensweise}

Mit aircrack-ng kann man mit einer Wordlist aus einem WLAN Dump das WPA Passwort herausfinden. Wir haben diese Wordlist verwendet: ftp://ftp.openwall.com/pub/wordlists/all.gz
\begin{lstlisting}
aircrack-ng -w wordlist i_dump02_6yiuXJ0MR8qOtysR.cap
\end{lstlisting}

\section{Task 3A}

\subsection{Wie lautet der Benutzername der als letztes das Projekt Sonnensturm geoeffnet hat?}

esunden

\subsection{Vorgehensweise}

Der Java Client für die Projektdatenbank enthält Passwort und Benutzername für die Anmeldung in Klartext. Dazu muss man das jar entpacken und dann den Bytecode der enthaltenen Dateien anschauen. Mit dem Befehl
\begin{lstlisting}
javap -c at/ac/tuwien/inso/introsec/ws2013/lab1/gui/ClientLocalLoginGui
\end{lstlisting}
sieht man 
\begin{lstlisting}
   13:  getfield    #81; //Field userTf1:Ljavax/swing/JTextField;
   16:  invokevirtual   #166; //Method javax/swing/JTextField.getText:()Ljava/lang/String;
   19:  getstatic   #169; //Field at/ac/tuwien/inso/introsec/ws2013/lab1/domain/Submarines.sideBound:Ljava/lang/String;
   22:  invokevirtual   #175; //Method java/lang/String.equalsIgnoreCase:(Ljava/lang/String;)Z
\end{lstlisting}

Das heißt der String im JTextField userTf1 wird mit dem String at/ac/tuwien/inso/introsec/ws2013/lab1/domain/Submarines.sideBound verglichen. Mit dem selben Befehl kann man sich dann den Bytecode für Submarines.class anschauen. Dort sieht man, dass der Variable sidebound der Wert von at/ac/tuwien/inso/introsec/ws2013/lab1/domain/Site.web zugewiesen wird. Man führt also javap -c auch für diese Klasse aus, kommt auf at/ac/tuwien/inso/introsec/ws2013/lab1/domain/Project.Nasa, von dort auf at/ac/tuwien/inso/introsec/ws2013/lab1/domain/Pictures.picasso, von dort auf at/ac/tuwien/inso/introsec/ws2013/lab1/domain/Hotel.spitzbergen und von dort auf den Benutzernamen DidIt.

Ähnlich kommt man auch auf das Passwort. In ClientLocalLoginGui sieht man
\begin{lstlisting}
   33:  getfield    #90; //Field passTf1:Ljavax/swing/JPasswordField;
   36:  invokevirtual   #179; //Method javax/swing/JPasswordField.getPassword:()[C
   39:  invokespecial   #183; //Method java/lang/String."<init>":([C)V
   42:  getstatic   #186; //Field at/ac/tuwien/inso/introsec/ws2013/lab1/conn/ServerWriter.mathListing:Ljava/lang/String;
   45:  invokevirtual   #175; //Method java/lang/String.equalsIgnoreCase:(Ljava/lang/String;)Z
\end{lstlisting}

Also der Wert vom JPasswordField passTf1 wird mit dem Wert von at/ac/tuwien/inso/introsec/ws2013/lab1/conn/ServerWriter.mathListing verglichen. ServerWriter.mathListing bekommt den Wert von at/ac/tuwien/inso/introsec/ws2013/lab1/gui/ClientWindowGui.windowID und das bekommt den Wert 4TehLulZ.
\\\\
Username:Passwort lautet also DitId:4TehLulZ
\\\\
Damit kann man sich in den Client einloggen. Mit Portforwarding kann man sich dann zum Server auf 192.168.10.22:54321 verbinden. Dort kann man in User und Passwort zB folgenden SQL Code injecten: \lstinline{'or 1=1;} und bekommt eine Auflistung von Dokumenten und Benutzernamen. Dort steht dann unter anderem Projekt Sonnensturm | esunden.



Wie funktioniert der Angriff auf das WLAN und wie könnten Sie das Netzwerk vor solchen Angriffen schützen?
Wie könnte der LAN Dump erstellt worden sein? Welche Sicherheitsmaßnahmen sollten das in Produktiv-Systemen verhindern?

\subsection{Sub-Ueberschrift 1}
Lorem ipsum dolor sit amet, consetetur sadipscing elitr, sed diam nonumy eirmod tempor invidunt ut labore et dolore magna aliquyam erat, sed diam voluptua. 

\subsection{Sub-Ueberschrift 2}
Lorem ipsum dolor sit amet, consetetur sadipscing elitr, sed diam nonumy eirmod tempor invidunt ut labore et dolore magna aliquyam erat, sed diam voluptua. At vero eos et accusam et justo duo dolores et ea rebum. 

\subsection{Sub-Ueberschrift 3}
Lorem ipsum dolor sit amet, consetetur sadipscing elitr, sed diam nonumy eirmod tempor invidunt ut labore et dolore magna aliquyam erat, sed diam voluptua. 

\section{Beispiele}

\subsection{Source Code formatieren}
Es folgen einige Beispiele wie Sourcecode in diesem Dokument formatiert und referenziert werden kann
(\hyperref[code:beispiel1]{siehe Listing~\ref*{code:beispiel1} auf Seite~\pageref*{code:beispiel1}} und \hyperref[code:beispiel2]{siehe Listing~\ref*{code:beispiel2} auf Seite~\pageref*{code:beispiel2}}).

Ebenso können kurzer Code oder kurze Befehle direkt in der Zeile in einem \lstinline{lstinline Block} mit typengleicher Schrift formatiert werden.

\lstinputlisting[caption=Example C/C++ file,label=code:beispiel1,style=c]{example.c}

\begin{lstlisting}[caption=Example bash script,label=code:beispiel2,style=simple]
#!/bin/bash
echo "Bash version ${BASH_VERSION}..."
for i in {0..10..2}
  do
     echo "Welcome $i times"
 done

echo "some very very very very very very very very very very very very very very very very very very very very long string"

exit 0;
\end{lstlisting}

\subsection{Bilder}

Es folgen einige Beispiele wie Bilder in diesem Dokument eingefuegt werden koennen
(\hyperref[fig:logo1]{siehe Abbildung~\ref*{fig:logo1} auf Seite~\pageref*{fig:logo1}}).

\begin{figure}[h!]
  \centering
  \fbox{
    \includegraphics[width=0.4\textwidth]{./imgs/esse-color.png}
  }
  \caption{ESSE Logo}
  \label{fig:logo1}
\end{figure}


%%%%%%%%%%%%%%%%%%%%%%%%%%%%%%%%%%%%%%%%%%%%%%%%%%%%%%%%%%%%%%%%%%%%%%
%
% DO NOT CHANGE THE FOLLOWING PART
%
%%%%%%%%%%%%%%%%%%%%%%%%%%%%%%%%%%%%%%%%%%%%%%%%%%%%%%%%%%%%%%%%%%%%%%

\end{document}


